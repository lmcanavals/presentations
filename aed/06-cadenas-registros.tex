\documentclass[aspectratio=169]{beamer}

\usetheme{upc}

\usepackage{etex}
\usepackage{pgf,pgfopts,pgfplots}
\usepackage{mathrsfs}
\usepackage{tikz-uml}

\pgfplotsset{compat=1.10}

\title{Algoritmos y Estructuras de Datos}
\subtitle{Cadenas, Registros, punteros, referencias, objetos, instancias}
\date{\the\year}
\institute{\href{http://www.upc.edu.pe}{Universidad Peruana de Ciencias Aplicadas}}

\begin{document}

\maketitle

\begin{frame}{Outline}
\tableofcontents
\end{frame}

%%%%%%%%%%%%%%%%%%%% Primera sección %%%%%%%%%%%%%%%%%%%%%%%%%%%%%%
\section{Logro}
\begin{frame}{Logro}
Al finalizar la sesión el estudiante implementará aplicaciones haciendo uso
de cadenas estilo C y C++, registros y objetos de manera experta.
\end{frame}

%%%%%%%%%%%%%%%%%%%% Segunda sección %%%%%%%%%%%%%%%%%%%%%%%%%%%%%%
\section{Cadenas}
\begin{frame}{Cadenas de caracteres}
\begin{block}{Definición}
Las cadena (del inglés string) son secuencias de caracteres y es un tipo de dato común en lenguajes de programación de alto nivel. Según la forma en la que se implementan pueden ser mutables o inmutables.
\end{block}
\end{frame}

\subsection{Cadenas estilo C}
\begin{frame}{char*}
Las cadenas en el lenguaje C se representan por medio de arreglos de tipo char los cuales se declaran de diversas formas.
\begin{itemize}
\item char*: como un puntero de tipo char.
\item char[100]: como un arreglo estático de tipo char.
\item const char*: como un puntero constante char.
\end{itemize}
\end{frame}
\begin{frame}{Consideraciones}
\begin{itemize}
\item Dada una cadena de longitud N, cadenas estilo C necesitan por lo menos N + 1 espacios de tipo char para almacenar dicha cadena dado que se utiliza el espacio adicional para el terminador de cadenas $'\backslash 0'$.
\item En caso de utilizar char*, es decir un puntero a char, para representar cadenas, no debemos olvidar reservar y liberar el espacio de memoria de manera responsable.
\item En caso de usar char[123], es importante también considerar el uso de operaciones de manera segura para no escribir más allá de los límites del arreglo.
\end{itemize}
\end{frame}
\begin{frame}{Operaciones}
\begin{itemize}
\item \#include <string.h> 
\item Tamaño: strlen(cadena)
\item Copiar: strcpy(d, s), strncpy(d, s, n)
\item Concatenar: strcat(d, s), strncat(d, s, n)
\item Comparar: strcmp(a, b), strncmp(a, b, n)
\item Buscar caracter: strchar(s, c)
\end{itemize}
\end{frame}

\subsection{Cadenas estilo C++}
\begin{frame}[fragile]{std::string}
Las cadenas en el lenguaje C++ se representan por medio de objetos de tipo std::string.
\begin{lstlisting}
#include <string>
#include <iostream>
using namespace std;
int main() {
  string cadena1;
  string cadena2(", mundo");
  cadena1 = "Hola";
  cout << cadena1 << cadena2 << endl;
  return 0;
}
\end{lstlisting}
\end{frame}
\begin{frame}{Operaciones}
\begin{itemize}
\item \#include <string> 
\item Tama??o: cadena.size()
\item Copiar: s = d;
\item Concatenar: s = s + d;, s += d;
\item Comparar: a == b;
\item Buscar: s.find(t);
\end{itemize}
\end{frame}
%%%%%%%%%%%%%%%%%%%% Tercera sección %%%%%%%%%%%%%%%%%%%%%%%%%%%%%%
\section{Registros o estructuras}
\begin{frame}[fragile]{Registros}
Los registros son una forma de agrupar variables bajo un nombre como si fueran de un tipo de dato.
Por ejemplo podemos crear un nuevo registro Alumno:
\begin{lstlisting}
struct Alumno {
  int edad;
  char tipo;
  double promedio;
}
\end{lstlisting}
\end{frame}
\begin{frame}[fragile]{Registros}
Podemos utilizar registro Alumno de la siguiente forma:
\begin{lstlisting}
Alumno a1;
a1.edad = 50;
a1.tipo = 'G';
a1.promedio = 4.55;
\end{lstlisting}
\end{frame}
\begin{frame}{Consideraciones}
\begin{itemize}
\item ¿Cuantos bytes ocupa en total una variable de tipo Alumno?
\item En C++ las clases y los registros solo se diferencian en que el tipo de acceso por defecto de las clases es privado y de los registros es público.
\item El uso de uno u otro depende del problema a tratar y las convenciones adoptadas por el equipo de desarrollo.
\end{itemize}
\end{frame}
%%%%%%%%%%%%%%%%%%%% Cuarta sección %%%%%%%%%%%%%%%%%%%%%%%%%%%%%%%
\section{Objetos, referencias, objetos e instancias}
\begin{frame}{Objetos}
\begin{block}{Instancias}
Un objeto o instancia es una porción de la memoria del computador que contiene un valor de un tipo definido como Clase y que por lo general está asociado a una variable llamada referencia.
\end{block}
\begin{block}{Punteros}
En el lenguaje C++ se recomienda manejar objetos de manera dinámica, en otras palabras usando punteros, creándolos con \alert{new} y liberándolos de memoria con \alert{delete}. \\
Es generalmente considerado una mala práctica usar objetos estáticos de tipo clase definida por el usuario. En otras palabras, esto no aplica a clases que forman parte de bibliotecas del propio lenguaje como por ejemplo std::string o std::vector, etc.
\end{block}
\end{frame}
%%%%%%%%%%%%%%%%%%%% última sección %%%%%%%%%%%%%%%%%%%%%%%%%%%%%%%
\section{Referencias}
\begin{frame}{Referencias}
\begin{thebibliography}{9}
\bibitem{cormen}
Thomas H. Cormen, Charles E. Leirserson, Ronald L. Rivest, Clifford Stein.
\textbf{Introduction to Algorithms}.
Third edition, The MIT Press, Cambridge, Massachusetts, 2009.
\bibitem{pppcppt}
Bjarne Stroustrup.
\textbf{Programming: Principles and practice using C++}.
Addison-Wesley, Upper Saddle River, NJ, Boston, 2009. Cap??tulo 19 secci??n 3, p. 656.
\bibitem{cstring}
cplusplus.com
\textbf{string.h}. http://www.cplusplus.com/reference/cstring/
\bibitem{string}
cplusplus.com
\textbf{string}. http://www.cplusplus.com/reference/string/
\bibitem{unders}
http://www.open-std.org/JTC1/SC22/WG14/www/docs/n1225.pdf
\end{thebibliography}
\end{frame}

\end{document}
