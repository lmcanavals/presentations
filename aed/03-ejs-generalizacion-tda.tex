\documentclass{article}

\usepackage[a4paper, margin=25mm]{geometry}

\usepackage[T1]{fontenc}
\usepackage[utf8x]{inputenc}
\usepackage{lmodern}

\title{Estructura de Datos y Algoritmos \\
  \large Unidad 01: Generalización y TDA}
\author{Todos los profesores}
\date{\the\year}

\begin{document}
\maketitle

\section{Funciones}

Implemente las siguientes funciones usando templates.
\begin{enumerate}
\item Encontrar el elemento mayor en un arreglo.
\item Encontrar el elemento menor en un arreglo.
\item Sumar todos los elementos de un arreglo.
\item Encontrar la moda de los elementos de un arreglo.
\item Ordenar un arreglo.
\item Eliminar el elemento en una posición de un arreglo.
\item Buscar un elemento en un arreglo.
\item Determinar si un elemento existe en un arreglo.
\item Calcular cuántas veces se repite un elemento X en un arreglo.
\end{enumerate}

\section{Clases}

Implemente las siguientes clases usando templates.
\begin{enumerate}
\item Punto, un par ordenado x, y.
\item Polígono el cual está definido por un conjunto de Puntos.
\item Curva de Bézier, definida por conjuntos de puntos.
  https://en.wikipedia.org/wiki/Bézier\_curve
\item Canvas, Compuesta por un conjunto de Curvas de Bezier.
\item Implemente un formulario que permita a un usuario dibujar curbas de Bezier ingresando los puntos correspondientes a un polígono.
\end{enumerate}

\end{document}