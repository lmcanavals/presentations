\documentclass{article}

\usepackage[a4paper, margin=25mm]{geometry}

\usepackage[T1]{fontenc}
\usepackage[utf8x]{inputenc}
\usepackage{lmodern}
\usepackage{listings}
\usepackage{xcolor}

\definecolor{UpcRed}{HTML}{FF1901} % rojo UPC
\definecolor{DarkGray}{HTML}{3E3E3E} % gris oscuro
\definecolor{Gray}{HTML}{626262}  % gris medio
\definecolor{LightGray}{HTML}{bbbbbb}  % gris claro

\lstset{
  frame=single,
  rulecolor=\color{LightGray},
  language=C++,
  aboveskip=3mm,
  belowskip=3mm,
  showstringspaces=false,
  columns=flexible,
  basicstyle={\small\ttfamily},
  numbers=none,
  numberstyle=\tiny\color{DarkGray},
  keywordstyle=\color{blue},
  commentstyle=\color{LightGray},
  stringstyle=\color{cyan},
  breaklines=true,
  breakatwhitespace=true,
  tabsize=2,
  literate=
  {á}{{\'a}}1 {é}{{\'e}}1 {í}{{\'i}}1 {ó}{{\'o}}1 {ú}{{\'u}}1
  {Á}{{\'A}}1 {É}{{\'E}}1 {Í}{{\'I}}1 {Ó}{{\'O}}1 {Ú}{{\'U}}1
  {à}{{\`a}}1 {è}{{\`e}}1 {ì}{{\`i}}1 {ò}{{\`o}}1 {ù}{{\`u}}1
  {À}{{\`A}}1 {È}{{\'E}}1 {Ì}{{\`I}}1 {Ò}{{\`O}}1 {Ù}{{\`U}}1
  {ä}{{\"a}}1 {ë}{{\"e}}1 {ï}{{\"i}}1 {ö}{{\"o}}1 {ü}{{\"u}}1
  {Ä}{{\"A}}1 {Ë}{{\"E}}1 {Ï}{{\"I}}1 {Ö}{{\"O}}1 {Ü}{{\"U}}1
  {â}{{\^a}}1 {ê}{{\^e}}1 {î}{{\^i}}1 {ô}{{\^o}}1 {û}{{\^u}}1
  {Â}{{\^A}}1 {Ê}{{\^E}}1 {Î}{{\^I}}1 {Ô}{{\^O}}1 {Û}{{\^U}}1
  {œ}{{\oe}}1 {Œ}{{\OE}}1 {æ}{{\ae}}1 {Æ}{{\AE}}1 {ß}{{\ss}}1
  {ç}{{\c c}}1 {Ç}{{\c C}}1 {ø}{{\o}}1 {å}{{\r a}}1 {Å}{{\r A}}1
  {€}{{\EUR}}1 {£}{{\pounds}}1 {ñ}{{\~n}}1 {Ñ}{{\~N}}1
}

\title{Estructura de Datos y Algoritmos \\
  \large Archivos}
\author{Todos los profesores}
\date{\the\year}

\begin{document}
\maketitle

\section{Archivos}
Considerando que solo debe usar operaciones de lectura y escritura de archivos binariol, implementar lo siguiente:
\begin{enumerate}
  \item Escriba un programa que permita listar un archivo binario llamado 
        "empleados.dat" que contiene la siguiente estructura:

\begin{lstlisting}
char nombre[10];
int edad;
double salario;
\end{lstlisting}

  \item Escriba un programa que permita listar al revés un archivo binario
        llamado "empleados.dat" que contiene la siguiente estructura:

\begin{lstlisting}
char nombre[10];
int edad;
double salario;
\end{lstlisting}

	El archivo \emph{no} deberá ser modificado.
  \item Escriba un programa que elimine de un archivo binario, todos los
        registros que tengan un código que tenga como primer dígito un
        número par. La estructura del archivo es como sigue:

\begin{lstlisting}
char codigo[8];
char nombre[20];
int edad;
\end{lstlisting}

  \item Escriba una función que busque en un archivo de texto una palabra
        determinada y de encontrarla devuelva el número de línea donde se
        encontró. Si no la encuentra devuelva -1. Considere que las líneas
        del archivo de texto tienen como máximo 1024 caracteres.

  \item Escriba una función que permita hacer una copia de un archivo de texto.
        La función deberá recibir el nombre del archivo de origen y el nombre
        del archivo destino.

  \item Escriba una función que permita hacer una copia de un archivo de
        texto. La función deberá recibir el nombre del archivo de origen y
        el nombre del archivo destino. El archivo destino deberá tener todas
        las letras vocales convertidas a la letra ‘a’.

	Ejemplo: arcoiris → arcaaras
  \item Escriba un programa que devuelva las siguientes estadísticas con
        respecto a un archivo de texto:
    \begin{enumerate}
      \item Cantidad de caracteres (letras + espacios + símbolos)
      \item Cantidad de letras
      \item Cantidad de espacios
      \item Cantidad de símbolos
      \item Cantidad de vocales
      \item Cantidad de mayúsculas
      \item Cantidad de minúsculas
      \item Frecuencia de letras.
    \end{enumerate}
	Los resultados deberán ser escritos en un archivo de texto que tenga
        el mismo nombre del archivo procesado y .estat.txt

	Ejemplo:

    \begin{enumerate}
      \item Archivo a procesar: Datos.txt
      \item Archivo de resultados: Datos.txt.estat.txt
    \end{enumerate}

  \item Escriba una función que permita contar la cantidad de palabras
        que existen en una cadena de caracteres. Considere que no existen
        2 espacios en blanco continuos.

        Escriba un programa que permita contar la cantidad de palabras
        que existen en un archivo de texto.

  \item Escriba un programa que permita procesar un archivo de texto que
        contiene un diccionario.

        El programa deberá pedirle al usuario que ingrese las primeras n
        letras de la palabra que desea buscar. Luego el programa mostrará
        todas las palabras que se encuentren en el archivo de texto que
        comiencen con las letras ingresadas. 

        Usted puede asumir que cada palabra del archivo de texto está en
        una línea nueva de la siguiente manera:

\begin{lstlisting}
abuelo
arbol
arbusto
barco
casa
zapato
\end{lstlisting}

	Si el usuario ingresara ar, el programa debería mostrar arbol
        y arbusto, sin embargo si solo ingresara la letra a, debería
        mostrar abuelo, arbol y arbusto.

	Puede asumir que el diccionario no tiene acentos y que el usuario
        tampoco ingresará acentos.

  \item Escriba un programa que permita administrar una lista de Alumnos
        utilizando POO y que permita almacenar y cargar la información
        desde un archivo.

        Cada alumno deberá tener los siguientes datos:
    \begin{enumerate}
      \item Edad
      \item Promedio
      \item Codigo
      \item Nombre
    \end{enumerate}

        Considere como máximo 100 alumnos y además que no pueden existir
        alumnos con códigos repetidos.

\end{enumerate}

\end{document}
