\documentclass[a4paper]{article}

\usepackage[a4paper, margin=25mm]{geometry}

\usepackage[utf8x]{inputenc}

\title{Estructura de Datos y Algoritmos \\
  \large Cadenas}
\author{Todos los profesores}
\date{\the\year}

\begin{document}
\maketitle

\section{Cadenas}
Implementar lo siguiente:
\begin{enumerate}
  \item Una función que permita contar todos los caracteres de una cadena hasta
    encontrar encontrar un caracter dado.
  \item Una función que permita contar todas las palabras dentro de una cadena.
    Considere que las palabras están separadas por un solo espacio en blanco.
  \item Una función que permita invertir una cadena. Por ejemplo dada la cadena
    "Planeta", debe producir "atenalP".
  \item Una función que permita determinar si una cadena es palindromo.
  \item Una función que permita contar cuantas veces se repite un caracter dado.
  \item Una función que permita partir una cadena en un arreglo de cadenas dado
    un caracter separador. (split)
  \item Una función que reciba un número y devuelva el equivalente en números
    romanos.
  \item Una función que dado el año mes y día, imprima dicha fecha en formato
    largo. Por ejemplo: dados 2015, 03 y 23, muestre 03 de Marzo del 2015.
  \item Una función que dado un número romano como cadena, devuelva el
    equivalente en entero.
  \item Una función que convierta una cadena a mayúsculas.
  \item Una función que convierta una cadena a minúsculas.
  \item Una función que convierta una cadena a formato título, por ejemplo para
    "las cadenas en c son interesantes" de "Las Cadenas En C Son Interesantes".
  \item Una función que dado un número entero, retorne dicho número en formato
    literal. Por ejemplo para 12 retorna "doce", para 2015 retorna "dos mil
    quince", etc.
  \item Una función que reemplace una palabra por otra palabra proporcionada
    dentro de una cadena.
\end{enumerate}


\end{document}
