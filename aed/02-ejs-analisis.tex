\documentclass{article}

\usepackage[a4paper, margin=25mm]{geometry}

\usepackage[T1]{fontenc}
\usepackage[utf8x]{inputenc}
\usepackage{lmodern}
\usepackage{listings}
\usepackage{xcolor}

\definecolor{UpcRed}{HTML}{FF1901} % rojo UPC
\definecolor{DarkGray}{HTML}{3E3E3E} % gris oscuro
\definecolor{Gray}{HTML}{626262}  % gris medio
\definecolor{LightGray}{HTML}{bbbbbb}  % gris claro

\lstset{
  frame=single,
  rulecolor=\color{LightGray},
  language=C++,
  aboveskip=3mm,
  belowskip=3mm,
  showstringspaces=false,
  columns=flexible,
  basicstyle={\small\ttfamily},
  numbers=none,
  numberstyle=\tiny\color{DarkGray},
  keywordstyle=\color{blue},
  commentstyle=\color{LightGray},
  stringstyle=\color{cyan},
  breaklines=true,
  breakatwhitespace=true,
  tabsize=2,
  literate=
  {á}{{\'a}}1 {é}{{\'e}}1 {í}{{\'i}}1 {ó}{{\'o}}1 {ú}{{\'u}}1
  {Á}{{\'A}}1 {É}{{\'E}}1 {Í}{{\'I}}1 {Ó}{{\'O}}1 {Ú}{{\'U}}1
  {à}{{\`a}}1 {è}{{\`e}}1 {ì}{{\`i}}1 {ò}{{\`o}}1 {ù}{{\`u}}1
  {À}{{\`A}}1 {È}{{\'E}}1 {Ì}{{\`I}}1 {Ò}{{\`O}}1 {Ù}{{\`U}}1
  {ä}{{\"a}}1 {ë}{{\"e}}1 {ï}{{\"i}}1 {ö}{{\"o}}1 {ü}{{\"u}}1
  {Ä}{{\"A}}1 {Ë}{{\"E}}1 {Ï}{{\"I}}1 {Ö}{{\"O}}1 {Ü}{{\"U}}1
  {â}{{\^a}}1 {ê}{{\^e}}1 {î}{{\^i}}1 {ô}{{\^o}}1 {û}{{\^u}}1
  {Â}{{\^A}}1 {Ê}{{\^E}}1 {Î}{{\^I}}1 {Ô}{{\^O}}1 {Û}{{\^U}}1
  {œ}{{\oe}}1 {Œ}{{\OE}}1 {æ}{{\ae}}1 {Æ}{{\AE}}1 {ß}{{\ss}}1
  {ç}{{\c c}}1 {Ç}{{\c C}}1 {ø}{{\o}}1 {å}{{\r a}}1 {Å}{{\r A}}1
  {€}{{\EUR}}1 {£}{{\pounds}}1 {ñ}{{\~n}}1 {Ñ}{{\~N}}1
}

\title{Estructura de Datos y Algoritmos \\
  \large Análisis de algoritmos}
\author{Todos los profesores}
\date{\the\year}

\begin{document}
\maketitle


\section{Ejercicio 01}
Suma y Promedio de números.

\begin{lstlisting}
int s = 0;
double prom = 0;
for (int i = 0; i < n; i++) {
  s = s + A[i];
  prom = s / (double) n;
}
printf("Suma:%d\nProm:%d", s, prom);
\end{lstlisting}

\section{Ejecicio 02}
Suma y Promedio de números (2.0).

\begin{lstlisting}
int s = 0;
double prom = 0;
for (int i = 0; i < n; i++) {
  s = s + A[i];
}
prom = s / (double)n;
printf("Suma:%d\nProm:%d", s, prom);
\end{lstlisting}

\section{Ejercicio 03}
Ordenamiento de N números del 1 al 100.

\begin{lstlisting}
for (int i = 0; i < n - 1; i++) {
  for (int k = i + 1; k < n; k++) {
    if (vec[i] > vec[k]) {
      int aux = vec[i];
      vec[i] = vec[k];
      vec[k] = aux;
    }
  }
}
\end{lstlisting}

\pagebreak
\section{Ejercicio 04}
Ordenamiento de N números del 1 al 100 (2.0).

\begin{lstlisting}
int frec[101] = {0};
for (int i = 0; i < n; i++) {
  frec[vec[i]]++;
}
int pos = 0;
for (int i = 0; i < 101; i++) {
  for (int k = 0; k < frec[i]; k++) {
    vec[pos] = i;
    pos++;
  }
}
\end{lstlisting}

\section{Ejercicio 05}
Algoritmo raro - Infinito.

\begin{lstlisting}
int i=0;
int sum=0;
while (i < 100) {
  if (i % 2 == 0) {
    for (int k = 0; k < n; k++) {
      sum += vec[i];
    }
  }
  else {
    for (int k=0; k<i; k++) {
      sum += vec[i];
    }
  }
}
\end{lstlisting}

\section{Ejercicio 06}
Factorial.

\begin{lstlisting}
int fact = 1;
for (int i = 2; i < n; i++) {
  fact *= i;
}
printf("Factorial: %d", fact);
\end{lstlisting}

\section{Ejercicio 07}
Buscar cadena de máximo 50 caracteres.

\begin{lstlisting}
int pos = -1;
for (int i = 0; i < n; i++) {
  if (strcmp(vec[i], cadBuscar) == 0) {
    pos = i;
    break;
  }
}
\end{lstlisting}

\section{Ejercicio 08}
Buscar el mayor.

\begin{lstlisting}
int pos = 0;
for (int i = 1; i < n; i++) {
  if (vec[i] > vec[pos])
    pos = i;
}
printf("El mayor es: %d", vec[pos]); 
\end{lstlisting}

\section{Ejercicio 09}
Otro algoritmo raro – El while es válido?.

\begin{lstlisting}
int max = 0;
for (int m = 0; m < n; m++) {
  int cont = 0;
  int k = m + 1;
  while (vec[m] <= vec[k]) {
    k = k + 1;
    cont++;
  }
  if (cont > max)
    max = cont;
}
printf("Maximo %d", max);
\end{lstlisting}

\section{Ejercicio 10}
Logarítmica.

\begin{lstlisting}
int i=1;
while (i < n) {
  if (vec[i] % 2 == 0)
  	i *= 3;
  else
  	i *= 2;
}
\end{lstlisting}

\section{Ejercicio 11}
Logarítmica (2.0).

\begin{lstlisting}
int i=1;
while (i < n) {
  if (vec[i] % 2 == 0)
  	n = n / 3;
  else
  	n = n / 2;
}
\end{lstlisting}

\pagebreak
\section{Ejercicio 12}
Búsqueda binaria – Arreglo ordenado.

\begin{lstlisting}
int inf = 0;
int sup = n - 1;
int pos = -1;

while ((pos == -1) && (sup >= inf)) {
  int medio = (inf + sup) / 2;
  if (arreglo[medio] == 80)
  	pos = medio;
  else if (arreglo[medio] < 80)
  	inf = medio + 1;
  else
  	sup = medio - 1;
}
\end{lstlisting}

\section{Ejercicio 13-22}
Implemente y analice las siguientes funciones.
\begin{enumerate}
\item Encontrar el número mayor en un arreglo de enteros
\item Ordenar un arreglo de números enteros
\item Eliminar el elemento en una posicion de un arreglo
\item Buscar un número en un arreglo
\item Calcular el factorial de N
\item Determinar si un número existe en un arreglo de enteros.
\item Calcular cuántas veces se repite un número X en un arreglo de enteros.
\item Sumar los dígitos de un número entero positivo.
\item Determinar si un número es primo o no.
\item Determinar la cantidad de primos que existen en un arreglo de enteros.
\end{enumerate}

\end{document}