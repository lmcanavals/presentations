\documentclass[aspectratio=169]{beamer}

\usetheme{upc}

\usepackage{etex}
\usepackage{pgf,pgfopts,pgfplots}
\usepackage{mathrsfs}
\usepackage{tikz-uml}

\pgfplotsset{compat=1.10}

\title{Programación Concurrente y Distribuida}
\subtitle{Fundamentos}
\date{2015}
\institute{\href{http://www.upc.edu.pe}{Universidad Peruana de Ciencias Aplicadas}}

\begin{document}

\maketitle

\begin{frame}
Presentación basada en el libro Java Concurrency in Practice de Brian Goetz.
\end{frame}

\begin{frame}{Outline}
\begin{minipage}{\textwidth}
\tableofcontents
\end{minipage}
\end{frame}

\section{Logro} % Sección 1 %%%%%%%%%%%%%%%%%%%%%%%%%%%%%%%%%%%%%%%%%%%%%%%%%%%
\begin{frame}{Logro}
Al finalizar la sesión el estudiante construye componentes y aplicaciones
considerando las implicaciones de la programación concurrente de manera
estratégica.
\end{frame}

\section{Thread safety} % Sección 2 %%%%%%%%%%%%%%%%%%%%%%%%%%%%%%%%%%%%%%%%%%%
\begin{frame}
\begin{itemize}
\item Atomicidad
\item Bloqueo
\item Liveness
\end{itemize}
\end{frame}

\begin{frame}{Thread safe}
\begin{block}{Definición}
Una clase es Thread safe si puede ser usada desde múltiples hilos sin causar
interacciones no deseadas.
\end{block}
\end{frame}

\begin{frame}{Atomicidad}
\begin{block}{Definición}
Una sola operación a la vez!
\end{block}
\begin{block}{Condición de carrera}
Situación que ocurre cuando 2 operaciones se intentan realizar al mismo tiempo
sobre un mismo recurso, provocando un resultado no deseado.
\end{block}
\end{frame}

\begin{frame}{Bloqueo}
\begin{block}{Definición}
Cuando la atomicidad por si sola no es suficiente para mantener la consistencia
de un dato, por ejemplo cuando se necesita mantener correlación entre múltiples
variables.\\
Formas comunes de implementar bloqueo es usando bloques sincronizados,
semáforos o mutex.
\end{block}
\end{frame}

\begin{frame}{Liveness}
\begin{block}{Definición}
Es la capacidad de una aplicación de ejecutarse de manera oportuna. A tiempo
o en algún momento.\\
Deadlock, starvation y livelock son problemas comunes
relacionados a Liveness.
\end{block}
\end{frame}

\section{Objetos compartidos} % Sección 3 %%%%%%%%%%%%%%%%%%%%%%%%%%%%%%%%%%%%%
\begin{frame}
\begin{itemize}
\item Visibilidad
\item Publicación y escape
\item Confinamiento al hilo
\item Inmutabilidad
\item Publicación segura
\end{itemize}
\end{frame}

\begin{frame}{Visibilidad}
\begin{block}{Definición}
No hay garantías de que el valor de una variable compartida por varios hilos,
sea pasado todos los hilos.\\
Por ejemplo, el compilador podría guardar en cache
el valor de una variable para optimizar el programa.\\
La solución es usar variables volátiles.
\end{block}
\end{frame}

\begin{frame}{Publicación y escape}
\begin{block}{Definición}
Es hacer accesible un objeto más allá de su alcance (scope), en otras palabras
existe una referencia al objeto en algún otro lugar del programa.\\
En caso este comportamiento ocurra sin que se haya previsto de esta manera, se
considera \alert{escape}.\\
Para asegurar una correcta publicación podría ser necesario el uso de bloqueo.
\end{block}
\end{frame}

\begin{frame}{Confinamiento al hilo}
\begin{block}{Definición}
Data que solo es accesible desde un hilo en particular, se considera confinada
y es la forma más sencilla de conseguir thead safety.
\end{block}
\end{frame}

\begin{frame}{Inmutabilidad}
\begin{block}{Definición}
Una variable es inmutable cuando su valor no puede ser modificado. Esto
asegura que problemas relacionados a atomicidad, simplemente no existan.
\end{block}
\end{frame}

\begin{frame}{Publicación segura}
\begin{block}{Definición}
\end{block}
\end{frame}

\section{Composición de objetos} % Sección 4 %%%%%%%%%%%%%%%%%%%%%%%%%%%%%%%%%%
\begin{frame}
\begin{itemize}
\item Diseño de clases thread safe
\item Confinamiento a instancia
\item Delegando thread safety
\item Agregando funcionalidad a clases thread safe
\item Políticas de sincronización
\end{itemize}
\end{frame}

\section{Puntos de partida} % Sección 5 %%%%%%%%%%%%%%%%%%%%%%%%%%%%%%%%%%%%%%%
\begin{frame}
\begin{itemize}
\item Colecciones sincronizadas
\item Colecciones concurrentes
\item Productores y consumidores
\item Métodos interrumpibles
\item Sincronizadores
\end{itemize}
\end{frame}

%%%%%%%%%%%%%%%%%%%
\section{Referencias} % Bibliografía %%%%%%%%%%%%%%%%%%%%%%%%%%%%%%%%%%%%%%%%%%
\begin{frame}{Referencias}
\begin{thebibliography}{9}
\bibitem{goetz}
GOETZ, Brian et al.
\textbf{Java Concurrency in Practice}.
Addison-Wesley, Upper Saddle River, NJ, 2006.
\end{thebibliography}
\end{frame}

\end{document}
