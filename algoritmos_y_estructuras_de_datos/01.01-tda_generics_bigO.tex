\documentclass[aspectratio=169]{beamer}

\usetheme{upc}

\usepackage[utf8x]{inputenc}
\usepackage{lmodern}

\title{Algoritmos y Estructuras de Datos}
\subtitle{Introducción a las Estructuras de Datos y Algoritmos}
\date{2015}
\institute{\href{http://www.upc.edu.pe}{Universidad Peruana de Ciencias Aplicadas}}

\begin{document}

\maketitle

\begin{frame}{Outline}
  \tableofcontents
\end{frame}

\section{Tipos de datos abstractos}

\begin{frame}{Computerphile}

\begin{block}{Dr. James Clewett}
\href{https://www.youtube.com/watch?v=p7nGcY73epw}{The Art of Abstraction}
\end{block}

\end{frame}

\begin{frame}{Definición}

\begin{block}{Abstracción}
Es el proceso de quitar características de algo para reducir su complejidad.
\end{block}

\begin{block}{Tipo de dato}
Un tipo particular de ítem, definido por los valores que éste puede tomar, el lenguaje de programación usado o las operaciones que se realizan sobre éste.
\end{block}

\end{frame}

\section{Generalización de tipos}
\section{Crecimiento de funciones}

\begin{frame}{Eficiencia}
  \begin{itemize}
    \item Una solución es eficiente si resuelve el problema dentro de sus limitaciones de recursos.
    \begin{itemize}
      \item Espacio (Estructura de datos)
      \item Tiempo (Algoritmos)
    \end{itemize}
    \item El costo de una solución es la cantidad de recursos que esta consume en tiempo y espacio.
  \end{itemize}
\end{frame}

\begin{frame}{...Eficiencia}
  \begin{itemize}
    \item Existen tres tipos generales de análisis de tiempo:
    \begin{itemize}
      \item Mejor caso,
      \item caso promedio y
      \item peor caso.
    \end{itemize}
    \item Existen otros tipos de análisis pero no nos concentraremos en ellos.
  \end{itemize}
\end{frame}

\begin{frame}{Selección de una estructura de datos}
  \begin{itemize}
    \item Analizar el tipo de input posible en el problema.
    \item Analizar el problema para determinar las limitaciones de recursos a los que la solución debe adaptarse.
    \item Determinar las operaciones básicas que deben ser soportadas. Evaluar las limitaciones de recursos para cada una de estas operaciones.
    \item Seleccionar la estructura de datos que mejor se adecúe a estos requerimientos.
  \end{itemize}
  Nota: Generalmente buscamos la solución más simple.
\end{frame}

\begin{frame}{Debemos preguntarnos...}
  \begin{itemize}
    \item Las inserciones se harán siempre en el primer registro, al final o en base a algún criterio?
    \item Se puede eliminar la información?
    \item Se procesan los datos en algún orden específico o se accede aleatoriamente?
  \end{itemize}
  Estas interrogantes nos ayudan a eliminar algunas posibilidades.
\end{frame}

\begin{frame}{Estructuras de datos}
  \begin{itemize}
    \item Cada Estructura de datos tiene costos y beneficios.
    \item Raramente una estructura de datos es mejor que otra en todas las circunstancias.
    \item Una estructura de datos requiere:
    \begin{itemize}
      \item Espacio para cada registro,
      \item tiempo para ejecutar cada operación básica y
      \item esfuerzo de programación.
    \end{itemize}
  \end{itemize}
\end{frame}

\begin{frame}{...Estructuras de datos}
  \begin{itemize}
    \item Cada problema tiene limitaciones de espacio y tiempo.
    \item Solo después de un cuidadoso análisis de las características del problema podremos decidir en la mejor estructura de datos para la solución.
    \item Ejemplo: En un banco
    \begin{itemize}
      \item Abrir una cuento - minutos.
      \item Transacciones - segundos.
      \item Cierre de cuentas - horas (trasnoche).
    \end{itemize}
  \end{itemize}
\end{frame}

\subsection{Fundamento matemático}

\begin{frame}{Eficiencia de algoritmos}
  \begin{itemize}
    \item Siempre existen diversas soluciones a un mismo problema, como escoger entre ellos?
    \item Suelen existir dos objetivos, muchas veces en conflicto, al construir el programa:
    \begin{itemize}
      \item diseñar un algoritmo que sea fácil de entender, codificar y mantener.
      \item Diseñar un algoritmo que haga uso eficiente de los recursos del computador.
    \end{itemize}
    \item El objetivo 1 es preocupación del ingeniero de software.
    \item El objetivo 2 es preocupación de las ciencias de computación, en el análisis de estructura de datos y algoritmos. Cómo medimos el costo de un algoritmo?
  \end{itemize}
\end{frame}

\begin{frame}{Cómo medir la eficiencia}
  \begin{itemize}
    \item Análisis asintótico de algoritmos.
    \begin{itemize}
      \item Recursos críticos.
      \item Factores que afectan el tiempo de ejecución.
      \item Para la mayoría de algoritmos, el tiempo de ejecución depende de los parámetros de entrada (size).
    \end{itemize}
  \end{itemize}
\end{frame}

\begin{frame}{Veamos un ejemplo}
  \begin{itemize}
    \item Después de realizar el análisis asintótico de dos algoritmos, \textcolor{red}{h} y \textcolor{green}{k}, obtuvimos las siguientes fórmulas de tiempo en base al número de datos n:
    \begin{itemize}
      \item \textcolor{red}{$h(n) = n^3 - 12n^2 + 20n + 110$}
      \item \textcolor{green}{$k(n) = n^3 + n^2 + 5n + 5$}
    \end{itemize}
    \item ¿Qué algoritmo es mejor?
  \end{itemize}
\end{frame}

\begin{frame}{}
  \begin{itemize}
    \item 
    \begin{itemize}
      \item 
    \end{itemize}
  \end{itemize}
\end{frame}

\subsection{Análisis de Algoritmos}

\begin{frame}{}
  \begin{itemize}
    \item 
    \begin{itemize}
      \item 
    \end{itemize}
  \end{itemize}
\end{frame}

\section{Referencias}
\begin{frame}{Referencias}
  \begin{thebibliography}{9}
    \bibitem{cormen}
    Thomas H. Cormen, Charles E. Leirserson, Ronald L. Rivest, Clifford Stein.
    \textit{Introduction to Algorithms}. Third edition, The MIT Press, Cambridge, Massachusetts, 2009.
  \end{thebibliography}
\end{frame}

\end{document}