\documentclass[a4paper]{article}

\usepackage[a4paper, margin=25mm]{geometry}

\usepackage[utf8x]{inputenc}

\title{Estructura de Datos y Algoritmos \\
  \large Unidad 01: Registros y Objetos}
\author{Todos los profesores}

\begin{document}
\maketitle

\section{Registros y objetos}
Implementar lo siguiente:
\begin{enumerate}
  \item Un registro llamado \emph{Point} con miembros enteros \emph{x} y
    \emph{y}.
  \item Una clase llamada \emph{Path} que contenga un arreglo de tipo
    \emph{Point} llamado \emph{points} y un entero llamado \emph{numPoints}.
    Además, métodos para agregar puntos, quitar puntos, recuperar el i-esimo
    punto.
  \item Una función que reciba un objeto Path e indique exáctamente cuanta
    memoria está utilizando.
  \item Una aplicación en consola con un menú de opciones que permita registrar
    puntos, eliminar puntos dado el índice, mostrar todos los puntos y mostrar
    el total de memoria utilizada en bytes. Si la cantidad es superior a 1024,
    debe automáticamente cambiar el formato a Kilobytes.
  \item Implemente una aplicación en módo gráfico que muestre un panel en blanco
    en el cual se puede agregar \emph{Points} (representados por circulos de 5
    pixels de diámetro) a un \emph{Path} al hacer click en el panel. Si solo un
    punto existe, este se mostrará solo en la pantalla, pero en caso se haya
    ingresado más de un punto, estos debe estar unidos por línea de manera
    secuencial, por ejemplo el primer y segundo punto están unidos, el segundo y
    tercer puntos están unidos tambien por otra línea, etc.
  \item Modificar el programa anterior de modo que sea posible también eliminar
    puntos al hacer click en un punto existente mientras se tiene presionada la
    tecla \emph{shift}. Además debe mostrar en todo momento en la barra de
    títulos el total de memoria usada por el objeto de tipo \emph{Path}.
\end{enumerate}

\end{document}
