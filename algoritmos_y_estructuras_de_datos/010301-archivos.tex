\documentclass[aspectratio=169]{beamer}

\usetheme{upc}

\usepackage{etex}
\usepackage{pgf,pgfopts,pgfplots}
\usepackage{mathrsfs}
\usepackage{tikz-uml}

\pgfplotsset{compat=1.10}

\title{Algoritmos y Estructuras de Datos}
\subtitle{Lectura y Escritura de Archivos}
\date{2015}
\institute{\href{http://www.upc.edu.pe}{Universidad Peruana de Ciencias Aplicadas}}

\begin{document}

\maketitle

\begin{frame}{Outline}
\tableofcontents
\end{frame}

%%%%%%%%%%%%%%%%%%%% Primera sección %%%%%%%%%%%%%%%%%%%%%%%%%%%%%%
\section{Logro}
\begin{frame}{Logro}
Al finalizar la sesión el estudiante implementará aplicaciones con datos
persistentes usando archivos de manera eficiente.
\end{frame}

%%%%%%%%%%%%%%%%%%%% Segunda sección %%%%%%%%%%%%%%%%%%%%%%%%%%%%%%
\section{Archivos}
\begin{frame}{A tomar en cuenta}
\begin{itemize}
\item Almacenar datos en variable o en arreglos es temporal.
\item Al terminar un programa, los datos almacenados se pierden.
\item Para conservar datos permanentemente se utilizan archivos.
\item Los archivos almacenan en dispositivos de almacenamiento secundario,
especialmente en discos.
\item Cuando se trabaja con archivos se utilizan buffers, almacenes temporales
de datos en memoria.
\item Las operaciones de saluda se hacen a través del buffer, y solo cuando el
buffer se llena se realiza la escritura en el disco y se vacía el buffer.
\end{itemize}
\end{frame}

\begin{frame}{Beneficios}
\begin{itemize}
\item Permite alcanzar gran cantidad de información en un medio rápidamente
accesible por la computadora.
\item Evita el hecho de tener que volver a leer datos.
\item Los datos almacenados pueden ser compartidos por más de un programa.
\end{itemize}
\end{frame}

\begin{frame}{Operaciones con archivos}
\begin{itemize}
\item Para operar con un archivo hay que referirse a él mediante el nombre.
\item Las operaciones básicas para usar un arhcivo son:
\begin{itemize}
\item Apertura, dándole nombre y otras especificaciones.
\item Escritura de datos.
\item Lectura de datos.
\item Modificación de datos.
\item Cierre.
\end{itemize}
\end{itemize}
\end{frame}

\begin{frame}{Cómo definir un archivo en C++}
\begin{itemize}
\item En primer lugar debemos incluir el archivo de cabecera \alert{fstream}
y declarar el uso del espacio de nombres \alert{std} para abreviar.
\item Ésta librería incluye 3 clases:
\begin{itemize}
\item ofstream: para operaciones de solo escritura.
\item ifstream: para operaciones de solo lectura.
\item fstream: para operaciones de lectura y escritura.
\end{itemize}
\end{itemize}
\end{frame}

\begin{frame}[fragile]{Apertura de un archivo}
\begin{itemize}
\item Existen 2 formas de abrir un archivo de cualquiera de los 3 tipos.
\item Cualquiera de las 2 formas requiere 2 parámetros: nombre de archivo y
modos de apertura. Los modos de apertura son opcionales.
\begin{itemize}
\item Usando el constructor:
\begin{lstlisting}
fstream archivo("nombre.archivo", ios::binary | ios::in);
\end{lstlisting}
\item Usando método open:
\begin{lstlisting}
ofstream archivo;
archivo.open("nombre.archivo", ios::binary);
\end{lstlisting}
\end{itemize}
\item Para verificar si el archivo fue abierto correctamente, se usa el método is\_open()
\end{itemize}
\end{frame}

\begin{frame}{Modos de apertura}
\begin{itemize}
\item ios::in: Operaciones de lectura.
\item ios::out: Operaciones de escritura.
\item ios::binary: Archivo binario.
\item ios::ate: Posiciona el cursor al final del archivo.
\item ios::app: Operaciones de escritura se realizan al final del archivo.
\item ios::trunc: Remplaza el archivo, sobre-escribe su contenido si existiera.
\end{itemize}
\end{frame}

\begin{frame}[fragile]{Cerrar un archivo}
Para cerrar un archivo se usa el método close().
\begin{lstlisting}
archivo.close();
\end{lstlisting}
\end{frame}

%%%%%%%%%%%%%%%%%%%% Tercera sección %%%%%%%%%%%%%%%%%%%%%%%%%%%%%%
\section{Archivos de acceso aleatorio}
\begin{frame}{Archivos binarios}
\begin{itemize}
\item También conocidos como archivos binarios.
\item Permiten acceder a cualquier parte del fichero en cualquier momento como si fueran vectores en memoria.
\item Las operaciones de lectura y (o) escritura pueden hacerse en cualquier punto del archivo.
\end{itemize}
\end{frame}

\begin{frame}[fragile]{Escritura de datos}
Para escribir se usa el método ostream\& write (const char* s, streamsize n)
\begin{block}{Parámetros}
\alert{s} es un puntero a un arreglo de por lo menos n bytes.\\
\alert{n} es el número de bytes a escribir.
\end{block}
Es importante destacar que se debe hacer casting a char* en caso el dato que se
desea escribir sea de otro tipo.
\begin{lstlisting}
long numero;
int arreglo[TAMANO];
archivo.write((char*)&numero, sizeof(long));
archivo.write((char*)arreglo, sizeof(int) * TAMANO);
\end{lstlisting}
\end{frame}

\begin{frame}[fragile]{Lectura de datos}
Para leer se usa el método istream\& read (char* s, streamsize n)
\begin{block}{Parámetros}
\alert{s} es un puntero a un arreglo de por lo menos n bytes.\\
\alert{n} es el número de bytes a escribir.
\end{block}
Es importante destacar que se debe hacer casting a char* en caso el dato que se
desea leer sea de otro tipo.
\begin{lstlisting}
long numero;
int arreglo[TAMANO];
archivo.read((char*)&numero, sizeof(long));
archivo.read((char*)arreglo, sizeof(int) * TAMANO);
\end{lstlisting}
\end{frame}

\begin{frame}{Operaciones complementarias}
\begin{itemize}
\item bad() retorna verdadero en caso de error de lectura y escritura. fail() igual
que el anterior pero además en caso de error de formato. eof() verdadero cuando
el cursor de lectura se encuentra al final del archivo. good() verdadero cuando
todos los anteriores retornan falso.
\item tellg() y tellp(), retornan la posición del cursor de lectura (get) y
escritura (put), respectivamente.
\item seekg() y seekp(), permiten ubicar los cursores de lectura y escritura en la
posición indicada. Pueden recibir una posición absoluta o relativa. En caso sea relativa
se usan las banderas ios::beg, ios::end y ios::cur para indicar el punto de referencia
a partir del cual se realizará el desplazamiento.
\end{itemize}
\end{frame}

\begin{frame}{Consideraciones}
\begin{itemize}
\item Es posible guardar datos de tipos de datos distintos, como en el ejemplo
anterior que se escribe un valor de tipo long y luego un arreglo de TAMANO 
elementos de tipo entero. 
\item En caso se guarden tipos de datos distintos, se recomienda hacerlo
en un orden determinado previamente establecido.
\item Se recomienda el uso de registros para simplificar el proceso de lectura
y escritura.
\end{itemize}
\end{frame}

%%%%%%%%%%%%%%%%%%%% Cuarta sección %%%%%%%%%%%%%%%%%%%%%%%%%%%%%%%
\section{Archivos de acceso secuencial}
\begin{frame}{Archivos de texto}
\begin{itemize}
\item La información sólo puede leerse y escribirse empezando desde el principio del archivo.
\item Son más sencillos de manejar, ya que requieren menos funciones, además son más rápidos.
\item Son útiles cuando sólo se quiere almacenar cierta información a medida que se recibe, y no interesa analizarla en el momento.
\end{itemize}
\end{frame}

\begin{frame}[fragile]{Escritura y lectura de texto}
\begin{block}{Just like cout y cin}
Para escribir y leer archivos de texto se usa la misma sintaxis usada con
los objetos cin y cout.
\end{block}
\begin{lstlisting}
ofstream archivo("archivo.txt");
archivo << "Hola mundo!" << endl;

ifstream archivo("datos.txt");
archivo >> numero;
\end{lstlisting}
\end{frame}

%%%%%%%%%%%%%%%%%%%% Última sección %%%%%%%%%%%%%%%%%%%%%%%%%%%%%%%
\section{Referencias}
\begin{frame}{Referencias}
\begin{thebibliography}{9}
\bibitem{cormen}
Thomas H. Cormen, Charles E. Leirserson, Ronald L. Rivest, Clifford Stein.
\textbf{Introduction to Algorithms}.
Third edition, The MIT Press, Cambridge, Massachusetts, 2009.
\bibitem{pppcppt}
Bjarne Stroustrup.
\textbf{Programming: Principles and practice using C++}.
Addison-Wesley, Upper Saddle River, NJ, Boston, 2009. Capítulo 19 sección 3, p. 656.
\bibitem{fsmream}
cplusplus.com
\textbf{fstream}. http://www.cplusplus.com/reference/fstream/
\bibitem{files}
cplusplus.com
\textbf{Input/Output with files}. http://www.cplusplus.com/doc/tutorial/files/
\end{thebibliography}
\end{frame}

\end{document}
